\documentclass[runningheads]{llncs}


\usepackage{graphicx}
\usepackage{amsmath,amssymb}
\usepackage{dratex}

%%%%%%%%%%%%%%%%%%%%%%%%%%%%%%%%%%%%%%%%%%%%%%%%%%%%%
\begin{document}
%%%%%%%%%%%%%%%%%%%%%%%%%%%%%%%%%%%%%%%%%%%%%%%%%%%%%

%\large


\title{An approximation algorithm for graph clustering with clusters of bounded sizes}

\author{
	Victor Il'ev\inst{1,2}        \and
	Svetlana Il'eva\inst{1}       \and
	Nikita Gorbunov\inst{2}
}

\authorrunning{V. Il'ev et al.}
\titlerunning{An approximation algorithm for graph clustering}

\institute{Dostoevsky Omsk State University, Omsk, Russia \\
	\email{iljev@mail.ru} \\
	\email{iljeva@mail.ru}
	\and Sobolev Institute of Mathematics SB RAS, Omsk, Russia \\
	\email{gorbunov\_nikita\_v@mail.ru}
}

\maketitle

\begin{abstract} 
In graph clustering problems, one has to partition the set of vertices 
of a graph into disjoint subsets (called clusters) so as to minimize 
the number of edges between clusters and the number of missing edges 
within clusters. 
We consider a version of the problem in which cluster sizes 
are bounded from above by a positive integer $s$.
This problem is NP-hard for any fixed $s \geqslant 3$. We propose a 
polynomial-time approximation algorithm for this version of the problem. 
Its performance guarantee is better than earlier known bounds. 

\keywords{graph clustering, approximation algorithm, performance guarantee.}

\end{abstract}



%%%%%%%%%%%%%%%%%%%%%%%%%%%%%%%%%%%%%%%%%%%%%%%%%%%%%%%%%%%
\section*{Introduction} 
%%%%%%%%%%%%%%%%%%%%%%%%%%%%%%%%%%%%%%%%%%%%%%%%%%%%%%%%%%%

Clustering is the problem of grouping an arbitrary
set of objects so that objects in each group 
are more similar to each other than to those in other clusters.
In other words, it is required to partition a given
set of objects into some pairwise nonintersecting
subsets (clusters) so that the sum of the number of similarities
between the clusters and the number of missing similarities
in the clusters would be minimum.

%There are several suggestions for the measure of similarity
%between two clusterings. 
One of the most visual formalizations of clustering 
is the graph clustering \cite{s05}. 
A version of this problem is known as the graph approximation problem.
In this problem, the similarity relation between objects is given by an
undirected graph whose vertices are in one-to-one correspondence
to the objects and whose edges connect similar objects. 
The goal is to partition the set of vertices 
of a graph into disjoint subsets (called clusters) so as to minimize 
the number of edges between clusters and the number of missing edges 
within clusters. 
The number of clusters can be given, bounded, or
undefined. The statements and interpretations of the graph
approximation problem can be found in 
\cite{aikt06,f71,f76,t74,z64}.

In the sequel, the graph approximation problem was repeatedly and
independently rediscovered and studied under various names 
(Correlation Clustering, Cluster Editing, etc. 
\cite{bbc04,bsy99,sst04}).

This paper is organized as follows.
Section 1 contains three well-known NP-hard versions of the graph
clustering problem and a brief survey of basic
results on computational complexity and approximability
of these problems. 
In Section 2, another problem of graph clustering 
with some bounds on the sizes of clusters is considered. 
This problem is NP-hard too. Polynomially solvable cases 
of the problem are discussed. 
In Section 3, we propose a polynomial-time approximation
algorithm with a tight performance guarantee $2(s-1)$
for the problem in which the cluster sizes are bounded
above by a given number $s \geqslant 3$. 
Thus, the problem of graph clustering with clusters of bounded sizes 
belongs to the class APX for every fixed $s$.
Section 4 contains some approximation algorithms with better performance 
guarantees for cases $s=3$ and $s=4$. 
Conclusion summarizes the results of the work.

%%%%%%%%%%%%%%%%%%%%%%%%%%%%%%%%%%%%%%%%%%%%%%%%%%%%%%%%%%%
\section{Graph clustering problems}
%%%%%%%%%%%%%%%%%%%%%%%%%%%%%%%%%%%%%%%%%%%%%%%%%%%%%%%%%%%

We consider only ordinary graphs, i.e., the graphs without loops
and multiple edges. An ordinary graph is called a {\it cluster graph} 
if its every connected component is a complete graph \cite{sst04}. 
Let ${\cal M}(V)$ be the family of all cluster graphs on the set of vertices
$V$, let ${\cal M}_k(V)$ be the family of all cluster graphs on the
vertex set $V$ having exactly $k$ connected components,
and let ${\cal M}_{\leqslant k}(V)$ be the family of all cluster graphs on
$V$ having at most $k$ connected components, ${2 \leqslant k
\leqslant |V|}$.

If $G_1 = (V, E_1)$ and $G_2 = (V, E_2)$ are ordinary graphs both
on the set of vertices $V$, then the distance $d(G_1,G_2)$ between
them is defined as
$$
   d(G_1,G_2)=|E_1 \Delta E_2|=|E_1 \setminus E_2|+|E_2 \setminus E_1|,
$$
i.e., $ d(G_1,G_2)$ is the number of distinct edges in $G_1$ and $G_2$. 

In the 1960s-1980s the following three graph approximation
problems were under study which can be considered as different
formalizations of the graph clustering problem 
\cite{f71,f76,if82,t74,z64}:

{\bf Problem $\bf GC$.} Given an ordinary graph  $G \! = \! (V, E)$, find
a graph ${M^* \! \in \! {\cal M}(V)}$ such that
$$
   d(G,M^*)=\min\limits_{M \in {\cal M}(V)} d(G,M).
$$

{\bf Problem $\bf GC_k$.} Given an ordinary graph  $G \! = \! (V, E)$ and
an integer $k$, ${2 \leqslant k \leqslant |V|}$, find a graph
${M^* \! \in \! {\cal M}_k(V)}$ such that
$$
   d(G,M^*)=\min\limits_{M \in {\cal M}_k(V)} d(G,M).
$$

{\bf Problem $\bf GC_{\leqslant k}$.} Given an ordinary graph  $G \! = \! (V, E)$ 
and an integer $k$, ${2 \leqslant k \leqslant |V|}$, find a graph 
${M^* \! \in \! {\cal M}_{\leqslant k}(V)}$ such that
$$
   d(G,M^*)=\min\limits_{M \in {\cal M}_{\leqslant k}(V)} d(G,M).
$$

The first theoretical results related to the graph
approximation problem were obtained in the 
last century.
In 1964, Zahn \cite{z64} 
studied Problem $\bf GC$ for the graphs of some
special form. 
In 1971, Fridman \cite{f71}
defined the first polynomially
solvable case of Problem $\bf GC$. He showed that Problem
$\bf GC$ for a graph without triangles can be reduced to constructing 
maximum matching in this graph. 
In 1986, K\v{r}iv\'{a}nek and Mor\'{a}vek \cite{km86}
showed that problem $\bf GC$ is NP-hard, but their article remained unnoticed.

In 2004, Bansal, Blum, and Chawla \cite{bbc04}
and independently Shamir, Sharan, and Tsur \cite{sst04}
proved again that Problem $\bf GC$ is NP-hard. 
In \cite{sst04} it was also proved that
Problem ${\bf GC_k}$ is NP-hard for every fixed $k \geqslant 2$.
In 2006, a more simple proof of this result was published 
by Giotis and Guruswamy \cite{gg06}.
In the same year, Ageev, Il'ev, Kononov, and Talevnin \cite{aikt06} 
independently proved that Problems ${\bf GC_2}$ and ${\bf GC_{\leqslant 2}}$ 
are NP-hard already for cubic graphs, 
and derived from this that both Problems $\bf GC_k$ and $\bf GC_{\leqslant k}$
are NP-hard for every fixed $k \geqslant 2$.

In \cite{bbc04},
a simple 3-approximation algorithm for Problem ${\bf GC_{\leqslant 2}}$ 
was proposed. In \cite{aikt06}, existence of a randomized polynomial-time 
approximation scheme for Problem ${\bf GC_{\leqslant 2}}$ was proved, 
and in \cite{gg06}, a randomized polynomial-time approximation
scheme was proposed for Problem ${\bf GC_{\leqslant k}}$ (for every fixed 
${k \geqslant 2}$). In 2008, pointing out that the complexity of the polynomial
time approximation scheme of \cite{gg06}
deprives it of practical
perspectives of using, Coleman, Saunderson and Wirth \cite{csw08}
proposed some 2-approximation algorithm for Problem ${\bf GC_{\leqslant 2}}$, 
applying a local search procedure to the feasible solutions 
obtained by the 3-approximation algorithm of \cite{bbc04}.
For Problem ${\bf GC_2}$, 
a 2-approximation algorithm was proposed by Il'ev, Il'eva, and Morshinin 
\cite{iim19}.
As regards to Problem ${\bf GC}$, it was shown in 2005 \cite{cgw05}
that Problem ${\bf GC}$ 
is APX-hard, and a 4-approximation algorithm was developed. In
2008, some 2.5-approximation algorithm for Problem ${\bf GC}$ was
presented in Ailon, Charikar, and Newman \cite{acn08}.
Finally, in 2015, ($2.06 - \varepsilon$)-appriximation algorithm
for Problem ${\bf GC}$ was proposed \cite{cmsy15}.

\medskip

%%%%%%%%%%%%%%%%%%%%%%%%%%%%%%%%%%%%%%%%%%%%%%%%%%%%%%%%%%%
\section{The graph clustering problem with clusters of bounded sizes}
%%%%%%%%%%%%%%%%%%%%%%%%%%%%%%%%%%%%%%%%%%%%%%%%%%%%%%%%%%%

In contrast to Problems ${\bf GC_k}$ and ${\bf GC_{\leqslant k}}$, 
where the restrictions are imposed on
the number of clusters, we now discuss the problem of graph clustering
with clusters of bounded sizes. 

Let ${\cal M}^{\leqslant s}(V)$ be the family of all cluster graphs on $V$ such that 
the size of each connected component is at most some integer $s$,
${2\leqslant s \leq|V|}$. 
We say that a cluster graph belongs to
${\cal M}^s(V)$ if the size of each of its connected components is
equal to $s$.

{\bf Problem $\bf GC^{\leqslant s}$.} Given an $n$-vertex graph $G = (V,E)$
and an integer $s$, ${2\leqslant s \leqslant n}$, find $M^* \in {\cal M}^{\leqslant s}(V)$ 
such that
$$
 d(G,M^*)=\min_{M \in {\cal M}^{\leqslant s}(V)} d(G,M).
$$

{\bf Problem $\bf GC^s$.} Given a graph $G = (V,E)$ such that $|V|
= sq$, where $s$ and $q$ are positive integers, find $M^* \in
{\cal M}^{s}(V)$ such that
$$
   d(G,M^*)=\min\limits_{M \in {\cal M}^{s}(V)} d(G,M).
$$

In \cite{bbc04}, in the proof of NP-hardness of Problem $\bf GC$ without
any constraints on the number and sizes of clusters, it is
actually shown that Problem ${\bf GC^{\leqslant 3}}$ is NP-hard. 
In \cite{in11}, the following is proved:

\begin{theorem} {\rm \cite{in11}} \label{t1} 
Problems $\bf GC^{\leqslant s}$ and 
$\bf GC^{s}$ are NP-hard for every fixed ${s \geqslant 3}$.
\end{theorem}

In \cite{in11}, the cases when the optimal solutions to Problems 
$\bf GC^{\leqslant s}$ and $\bf GC^s$ can be found in polynomial time 
are also considered:

\begin{theorem} {\rm \cite{in11}} \label{t2} 
Problems ${\bf GC^{\leqslant 2}}$ and ${\bf GC^2}$ are polynomially solvable.
Problem ${\bf GC^{\leqslant 3}}$ on graphs without triangles is
polynomially solvable.
\end{theorem}

In \cite{iin16}, it is shown that the latter result can be generalized 
to the case of an arbitrary $s$. 

\begin{theorem} {\rm  \cite{iin16}} \label{t3} 
Problem ${\bf GC^{\leqslant s}}$ on graphs without triangles is
polynomially solvable for all $s \geqslant 2$.
\end{theorem}

\medskip

%%%%%%%%%%%%%%%%%%%%%%%%%%%%%%%%%%%%%%%%%%%%%%%%%%%%%%%%%%%
\section{An approximation algorithm for Problem ${\bf GC^{\leqslant s}}$}
%%%%%%%%%%%%%%%%%%%%%%%%%%%%%%%%%%%%%%%%%%%%%%%%%%%%%%%%%%%

In \cite{iin16}, for Problem $\bf GC^{\leqslant s}$ ($s \geqslant 3$) a
polynomial time approximation algorithm with tight performance
guarantee $\lfloor\frac{(s-1)^2}{2}\rfloor+1$ is proposed. 
In this section, we offer a polynomial-time approximation algorithm 
with the performance guarantee $2(s-1$) for this problem.

\medskip

{\bf Algorithm A}
\\
\textbf{Input:} An arbitrary graph $G = (V,E)$, an integer $s$, 
$3 \leqslant s \leqslant |V|$.
\\
\textbf{Output:} A cluster graph $M = (V_M,E_M) \in {\cal M}^{\leqslant s}$ -- 
approximate solution to problem ${\bf GC^{\leqslant s}}$.
\\ 
\textbf{Iteration $0$.} $V_M \leftarrow \emptyset$, $E_M \leftarrow \emptyset$, 
$V' \leftarrow \emptyset$, $E' \leftarrow E$, $G' \leftarrow (V',E')$. 
Go to iteration $1$.
\\ 
\textbf{Iteration $i$} ($i \geqslant 1 $). Constructing $i$-th cluster $K^i$.
 
\textbf{Step 1.} If $V'=\emptyset$, then \textbf{END}.
Else 
%if $V' \neq \emptyset$, then 
select an arbitrary vertex  ${v \in V'}$,\\ $K^i \leftarrow \{v\}$.
 
\textbf{Step 2.} If $|K^i|<s$, then go to step $3$, else go to step $5$.
 
\textbf{Step 3.} If there exists $ u \in V^{'}$ such that $uv \in E^{'}$ for any $v \in K^i $, then go to step $4$, else go to step $5$. 
 
\textbf{Step 4.} $K^i \leftarrow K^i\cup \{u\}$, go to step 2.

\textbf{Step 5.} $V_M \leftarrow V_M \cup K^i$, 
$E_M \leftarrow E_M \cup \{uv \ | \ u,v \in K^i\}$, 
$V^{'} \leftarrow V^{'} \setminus K^i$, 
$E^{'} \leftarrow E^{'} \setminus  \{uv \ | \ \{u,v\} \cap K^i \neq \emptyset\}$.
Go to iteration $i+1$.

\medskip

\textbf{Comments.}

\textbf{Step 0.} $V^{'}$ is the set of undistributed by clusters 
vertices of the graph $G$, 
$G'$ is the subgraph of $G$ induced by the set $V'$.

\textbf{Step 1.} Start constructing a new cluster $K^i$ of the graph $M$.

\textbf{Steps $2-4$.} Increasing the cluster.

\textbf{Step $5$.} Put in $M$ the clique induced by the set $K^i$ 
and delete it from the graph $G'$ with all edges incident to its vertices.


\begin{theorem} \label{t4}
  Let $G=(V,E)$ be an arbitrary graph, $s$ be an integer, $3 \leqslant s \leqslant |V|$.
Then
  \begin{equation}\label{e2}
  \frac{d(G,M)}{d(G,M^*)} \leqslant 2(s - 1),
  \end{equation}
where $M^* = (V_{M^*},E_{M^*})$ is an optimal solution to problem 
${\bf GC^{\leqslant s}}$ 
on the graph $G$, $M = (V_M,E_M)$ is the cluster graph constructed 
by Algorithm A. 
\end{theorem}

\textbf{Proof.}
1) By the definition, $d(G,M^*) = |E_G \setminus E_{M^*}| + |E_{M^*} \setminus E_G|$.
Let $E_G \setminus E_{M^*} = E_0 \cup E_1$, where
 
$E_0$ is the set of edges of the graph $G$ that are not placed neither 
in $M^*$, nor in $M$ by Algorithm A;
 
$E_1$ is the set of edges of the graph $G$ that are not placed in $M^*$, 
but are placed in $M$ by Algorithm A.

Denote $E_2 = E_{M^*} \setminus E_G$. Then $d(G,M^*) = |E_0| + |E_1| + |E_2|$.

2) By constructing, the graph $M$ is a subgraph of the graph $G$, 
therefore $d(G,M) = |E_G \setminus E_M|$.
Write the difference $E_G \setminus E_M$ in the following form: 
$E_G \setminus E_M = E_0 \cup \tilde{E}$, where $\tilde{E}$ is the set 
of edges of G not included in $M$, but placed in $M^*$. 
Edges of the set $\tilde{E}$ are edges of the graph G between clusters (cliques) 
of the graph $M$, hence any edge of the set $\tilde{E}$ has the form 
$e = uv$, where $v \in K^i$, $u \in K^j$ for some $i,j$, $i < j$ 
(see step 5).

3) Show that any edge of $\tilde{E}$ either is adjacent to an edge 
of the set $E_1$ in the graph $G$, or is adjacent to an edge of the 
set $E_2$ in the graph $M^*$.

Consider an arbitrary edge $e = uv$, where $v \in K^i$, $u \in K^j$, $i < j$. 
Since $i < j$, hence $u \in V'$ by constructing the cluster $K^i$ (on step 1 
of iteration $i$). At the same time, the vertex $u$ isn't included in the cluster 
$K^i$. It is possible only in two cases:

a) $|K^i| = s$. Then $|K^i \cup \{u\}| = s + 1$, therefore there exists 
at least one vertex $w \in K^i$ such that $w$ and $v$ are 
in different clusters of the graph $M^*$. And since by constructing 
$K^i$ is the vertex set of a clique of the graph $G$, hence vertices 
$v$ and $w$ are adjacent in $G$. Therefore, $vw \in E_1$.

b)$|K^i| < s$, but there exists $w \in K^i$ such that $uw \notin E$. If 
$w$ is placed in the same cluster of the graph $M^*$ as $u$, then $uw\in E_2$. 
Else, if $w$ and $u$ are in different clusters of the graph $M^*$, 
then vertices $w$ and $v$ are also in different clusters of $M^*$. 
As in case a) we get that $vw \in E_1$.

So, it is shown that any edge from $\tilde{E}$ either is adjacent to
an edge of the set $E_1$ in $G$, or is adjacent to an edge of the set 
$E_2$ in $M^*$. 
Hence $\tilde{E} = \tilde{E}_1 \cup \tilde{E}_2$, where
$$
\tilde{E}_1 = \{uv \in \tilde{E} \ | \ \exists vw\in E_1\}, 
\tilde{E}_2 = \{uv \in \tilde{E} \ | \ \exists uw\in E_2\}
$$
(it is possible that $\tilde{E}_1 \cap \tilde{E}_2 \neq \emptyset$).

4) Note, that for every edge $uv \in E_1$ there exist at most $2(s-1)$ edges
of $\tilde{E}_1$ incident to $u$ or $v$ in $G$, 
and for every edge $uv \in E_2$ there exist at most $2(s-2)$ edges
of $\tilde{E}_2$ incident to $u$ or $v$ in $G$.
Hence 
$$
d(C,M) \leqslant |E_0| + |\tilde{E}_1| + |\tilde{E}_2| 
\leqslant |E_0| + 2(s-1)|E_1| + 2(s-2)|E_2|
$$ $$ 
\leqslant |E_0|+(2s-1)(|E_1| + |E_2|).
$$
Therefore,
$$
\frac{d(G,M)}{d(G,M^*)} 
\leqslant \frac{|E_0| + 2(s-1)(|E_1| + |E_2|)}{|E_0| + |E_1| + |E_2|} 
\leqslant \frac{2(s-1)(|E_1|+|E_2|)}{|E_1| + |E_2|} = 2(s-1).
$$
Theorem \ref{t4} is proved.

\medskip

{\bf Corollary 1.} Problem ${\bf GC^{\leqslant s}}$ belongs to the class APX
for every fixed $s \geqslant 3$.

\medskip

By the following assertion, bound (\ref{e2}) is tight for any value of $s$.

\medskip

{\bf Remark 1.} For every $s \geqslant 3$, there exists a graph $G_s$
such that
\begin{equation}\label{e8}
d(G_s,M) = 2(s-1)d(G,M^*).
\end{equation}

\medskip

The graph $G_s$ has $2s$ vertices and consists of
the two cliques $K_s$ connected by one edge (bridge). Fig. 1 shows an
example for the case $s = 4$.


\begin{figure}[htb]

\Draw%
\Scale(1,1)

\LineAt(30,20,60,20)
\LineAt(30,-20,60,-20)
%\LineAt(0,-20,60,20)

\LineAt(30,-20,60,20)

\LineAt(30,20,60,-20)
\LineAt(30,20,30,-20)\LineAt(60,20,60,-20)
\MoveTo(30,20)\PaintCircle(2)\MoveTo(30,-20)\PaintCircle(2)
\MoveTo(60,20)\PaintCircle(2)\MoveTo(60,-20)\PaintCircle(2)
\MoveTo(30,27)\Text(--$2$--)\MoveTo(30,-27)\Text(--$7$--)
\MoveTo(60,27)\Text(--$6$--)\MoveTo(60,-27)\Text(--$8$--)

%\LineAt(30,-25,0,-25)

\LineAt(0,20,-30,20)
\LineAt(0,-20,-30,-20)
\LineAt(0,-20,-30,20)\LineAt(0,20,-30,-20)
\LineAt(0,20,0,-20) \LineAt(-30,20,-30,-20)
\MoveTo(0,20)\PaintCircle(2)\MoveTo(0,-20)\PaintCircle(2)
\MoveTo(-30,20)\PaintCircle(2)\MoveTo(-30,-20)\PaintCircle(2)
\MoveTo(0,27)\Text(--$1$--)\MoveTo(0,-27)\Text(--$5$--)
\MoveTo(-30,27)\Text(--$3$--)\MoveTo(-30,-27)\Text(--$4$--)

\LineAt(60,20,-20,20)%\LineAt(60,-20,-20,-20)

\EndDraw \caption{The graph $G_4$}\label{f1}
\end{figure}

\begin{figure}[htb]
\Draw%
\Scale(1,1)
\hspace*{-5mm}

\LineAt(30,20,60,20)\LineAt(30,-20,60,-20)\LineAt(30,-20,60,20)\LineAt(30,20,60,-20)
\LineAt(30,20,30,-20)\LineAt(60,20,60,-20)
\MoveTo(30,20)\PaintCircle(2)\MoveTo(30,-20)\PaintCircle(2)
\MoveTo(60,20)\PaintCircle(2)\MoveTo(60,-20)\PaintCircle(2)
\MoveTo(30,27)\Text(--$2$--)\MoveTo(30,-27)\Text(--$7$--)
\MoveTo(60,27)\Text(--$6$--)\MoveTo(60,-27)\Text(--$8$--)

\LineAt(0,20,-30,20)\LineAt(0,-20,-30,-20)\LineAt(0,-20,-30,20)\LineAt(0,20,-30,-20)
\LineAt(0,20,0,-20)\LineAt(-30,20,-30,-20)
\MoveTo(0,20)\PaintCircle(2)\MoveTo(0,-20)\PaintCircle(2)
\MoveTo(-30,20)\PaintCircle(2)\MoveTo(-30,-20)\PaintCircle(2)
\MoveTo(0,27)\Text(--$1$--)\MoveTo(0,-27)\Text(--$5$--)
\MoveTo(-30,27)\Text(--$3$--)\MoveTo(-30,-27)\Text(--$4$--)

\MoveTo(15,-43) \Text(--The graph $M^*$--)
%-----------
\hspace*{1cm}

\LineAt(120,20,150,20)
\LineAt(90,20,120,-20) \LineAt(180,20,150,-20)
\LineAt(90,-20,120,-20) \LineAt(180,-20,150,-20)
\LineAt(90,20,90,-20) \LineAt(180,20,180,-20)

\MoveTo(150,20)\PaintCircle(2)\MoveTo(150,-20)\PaintCircle(2)
\MoveTo(180,20)\PaintCircle(2)\MoveTo(180,-20)\PaintCircle(2)
\MoveTo(150,27)\Text(--$2$--)\MoveTo(150,-27)\Text(--$7$--)
\MoveTo(180,27)\Text(--$6$--)\MoveTo(180,-27)\Text(--$8$--)

\MoveTo(120,20)\PaintCircle(2)\MoveTo(120,-20)\PaintCircle(2)
\MoveTo(90,20)\PaintCircle(2)\MoveTo(90,-20)\PaintCircle(2)
\MoveTo(120,27)\Text(--$1$--)\MoveTo(120,-27)\Text(--$5$--)
\MoveTo(90,27)\Text(--$3$--)\MoveTo(90,-27)\Text(--$4$--)

\MoveTo(135,-43) \Text(--The graph $M$--)

%\medskip

\EndDraw\caption{The graphs $M^*$ and $M$}\label{f2}
\end{figure}


As we see in Fig. 2, $d(G_4,M^*)=1$, $d(G_4,M)=6$, and equality
(\ref{e8}) holds.

Clearly, $d(G_s,M^*)=1$. In the worst case the cluster graph $M$
for $G_s$ found by Algorithm A consists of a clique $K_2$ and two
cliques $K_{s-1}$, and hence $d(G_s,M) =2(s-1)$. 
Thus, we obtain (\ref{e8}).

\medskip

%%%%%%%%%%%%%%%%%%%%%%%%%%%%%%%%%%%%%%%%%%%%%%%%%%%%%%%%%%%
\section{Cases $s=3$ and $s=4$}
%%%%%%%%%%%%%%%%%%%%%%%%%%%%%%%%%%%%%%%%%%%%%%%%%%%%%%%%%%%

It is easy to see that 
$$
  \Bigg\lfloor\frac{(s-1)^2}{2}\Bigg\rfloor + 1 < 2(s-1) 
  \ \mbox{ for } s=3, 4,
$$ 
and 
$$ 
  \Bigg\lfloor\frac{(s-1)^2}{2}\Bigg\rfloor + 1 \geqslant 2(s-1) 
  \ \mbox{ for all } s \geqslant 5.
$$

In the cases $s=3$ and $s=4$ for Problem ${\bf GC^{\leqslant s}}$
the following polynomial-time approximation algorithms can be used 
\cite{iin16}.  
Algorithm A1 accepts as input an arbitrary connected graph $G$,
and Algorithm A2 accepts as input an arbitrary graph $G$
and uses Algorithm A1 as a procedure.

\medskip

{\bf Algorithm $A1$}
\\
\textbf{Input}: A connected graph $G = (V,E)$.
\\
\textbf{Output}: A cluster graph $M  \in {\cal M}^{\leqslant s}(V)$.

\textbf{Step 1.} Remove from $G$ all bridges and denote the resulting graph 
by $M_1$. Go to step 2.

\textbf{Step 2.} Construct a cluster graph $M_2\in {\cal M}^{\leqslant 2}(V)$: 
find a maximum matching in $G$; the found
matching forms bimodal components of the cluster graph $M_2$, 
and the vertices that are not included
in the matching form its trivial components. Go to step 3.

\textbf{Step 3.} If $M_1 \in {\cal M}^{\leqslant s}(V)$ and 
$d(G,M_1)\leqslant d(G,M_2)$, then 
$M \leftarrow M_1$, else $M \leftarrow M_2$.
%put $M = M_1$; otherwise, $M = M_2$.

\textbf{End.}

\medskip

%In this section we show that (6) remains valid for a disconnected graph.
%We describe the approximation
%algorithm for Problem $A^{1,p}$ on an arbitrary graph $G$:

{\bf Algorithm $A2$}
\\
\textbf{Input:} An arbitrary graph $G = (V,E)$, where $G_i = (U_i,E_i)$ 
is the $i$th connected component of $G$,
$i = 1, \dots , k$ for some $k \in \{1, \dots , |V |\}$.
\\
\textbf{Output}: A cluster graph $M  \in {\cal M}^{\leqslant s}(V)$.

\textbf{Step 1.} For every $G_i$ build a graph $M_i \in {\cal}M^{\leqslant s}(U_i)$, 
$i = 1, \dots , k$, by Algorithm A1. Go to step 2.

\textbf{Step 2.} $M \leftarrow \cup_{i =1}^k M_i$.

\textbf{End.}

\begin{theorem} \label{t5}
Algorithm A2 is the 3-approximation algorithm for Problem ${\bf GC^{\leqslant 3}}$
and 5-approximation algorithm for Problem ${\bf GC^{\leqslant 4}}$.
\end{theorem}

Indeed, it is proved in \cite{iin16} that for every graph $G$
$$
  d(G,M) \leqslant d(G,M^*) \Bigg\lfloor\frac{(s-1)^2}{2}\Bigg\rfloor+1,
$$
where $M \in {\cal M}^{\leqslant s}(V)$ is the cluster graph built 
by 
%Algorithm A1 or 
Algorithm A2, and $M^*$ is an optimal solution
to Problem ${\bf GC^{\leqslant s}}$ on $G$.
We have $\lfloor\frac{(s-1)^2}{2}\rfloor+1 =3$ \ for $s=3$
and $\lfloor\frac{(s-1)^2}{2}\rfloor+1 =5$ \ for $s=4$.

So, in the cases $s=3$ and $s=4$ bounds 3 and 5 are better than performance guarantee (\ref{e2})
of Algorithm A.
%(? So, in cases $s=3$ and $s=4$ Algorithms A1 and A2 are more preferable
%than Algorithm A ?).

\medskip

%%%%%%%%%%%%%%%%%%%%%%%%%%%%%%%%%%%%%%%%%%%%%%%%%%%%%%%%%%%
\section*{Conclusion}
%%%%%%%%%%%%%%%%%%%%%%%%%%%%%%%%%%%%%%%%%%%%%%%%%%%%%%%%%%%

A version of the graph clustering problem is
considered. In this version sizes of all clusters don't exceed a
given positive integer $s$. This problem is NP-hard for every
fixed $s \geqslant 3$. A new polynomial-time approximation algorithm 
is presented and a bound on worst-case behaviour of this algorithm 
is obtained. 
Some 3-approximation and 5-approximation algorithms
are proposed for the cases $s=3$ and $s=4$, respectively. 
Performance guarantees of these algorithms are better 
than ones of earlier presented approximation algorithms.

\bigskip

\bibliographystyle{splncs04}
%%%%%%%%%%%%%%%%%%%%%%%%%%%%%%%%%%%%%%%%%%%%%%%%%%%%%%%%%%%%%%%
%%%%%%%%%%%%%%%%%%%%%%%%%%%%%%%%%%%%%%%%%%%%%%%%%%%%%%%%%%%%%%%
\begin{thebibliography}{50}
%%%%%%%%%%%%%%%%%%%%%%%%%%%%%%%%%%%%%%%%%%%%%%%%%%%%%%%%%%%%%%%
%%%%%%%%%%%%%%%%%%%%%%%%%%%%%%%%%%%%%%%%%%%%%%%%%%%%%%%%%%%%%%%

\bibitem{aikt06} 
Ageev, A.A., Il'ev, V.P., Kononov, A.V., Talevnin, A.S.:
Computational complexity of the graph approximation problem.
Diskretnyi Analiz i Issledovanie Operatsii. Ser.~1. \textbf{13}(1),
3--11 (2006) (in Russian). English transl. in: J. of Applied and
Industrial Math. \textbf{1}(1), 1--8 (2007)
%
\bibitem{acn08}
Ailon, N., Charikar, M., Newman, A.: Aggregating Inconsistent Information:
Ranking and Clustering. J. of the ACM. \textbf{55}(5), 1--27 (2008)
%
\bibitem{bbc04}
Bansal, N., Blum, A., Chawla, S.: Correlation clustering. Machine
Learning. \textbf{56}, 89--113 (2004)
%
\bibitem{bsy99}
Ben-Dor, A., Shamir, R., Yakhimi, Z.: Clustering gene expression patterns.
J. of Computational Biology. \textbf{6}(3--4), 281--297 (1999)
%
\bibitem{cgw05}
Charikar, M., Guruswami, V., Wirth, A.: Clustering with
qualitative information. J. of Computer and System Science. 
\textbf{71}(3), 360--383 (2005)
%
\bibitem{cmsy15}
Chawla, S., Makarychev, K., Schramm, T., Yaroslavtsev, G.: Near Optimal
LP Algorithm for Correlation Clustering on Complete and Complete k-partite
Graphs. STOC '15 Symposium on Theory of Computing: ACM New York.
219-228 (2015)
%
\bibitem{csw08}
Coleman, T., Saunderson, J., Wirth, A.:
A local-search 2-approximation for 2-correlation-clustering.
Lecture Notes in Computer Science. \textbf{5193}, 308--319 (2008)
%
\bibitem{f71}
Fridman, G.\v{S}.: A Graph Approximation Problem, in Upravlyaemye
Sistemy. Izd. Inst. Mat., Novosibirsk.  \textbf{8}, 73--75 (1971) (in Russian)
%
\bibitem{f76}
Fridman, G.\v{S}.: Investigation of a Classifying Problem on Graphs. 
Methods of Modelling and Data Processing (Nauka, Novosibirsk).
147-177 (1976) (in Russian)
%
\bibitem{gg06}
Giotis, I., Guruswami, V.: Correlation clustering with a
fixed number of clusters. Theory of Computing.
\textbf{2}(1), 249--266 (2006)
%
\bibitem{if82}
Il'ev, V.P., Fridman, G.\v{S}.: On the Problem of Approximation
by Graphs with a Fixed Number of Components.
Dokl. Akad. Nauk SSSR. \textbf{264}(3), 533--538 (1982) (in Russian).
English transl. in: Sov. Math. Dokl. \textbf{25}(3), 666--670 (1982)
%
\bibitem{in11}
Il'ev, V.P., Navrotskaya, A.A.: Computational Complexity 
of the Problem of Approximation by Graphs with Connected Components
of bounded Size. Prikl. Diskretn. Mat. \textbf{3}(13), 80--84 (2011)
%
\bibitem{iin16}
 Il'ev, V.P., Il'eva, S.D., Navrotskaya, A.A.: Graph Clustering 
with a Constraint on Cluster Sizes. Diskretn. Anal. Issled. Oper.
\textbf{23}(3), 50-20 (2016) (in Russian).
English transl. in: J. Appl. Indust. Math. \textbf{10}(3), 341--348 (2016)
%
\bibitem{iim19}
Il'ev, V., Il'eva, S., Morshinin, A.: A 2-approximation
algorithm for the graph 2-clustering problem.
Lecture Notes in Comput. Sci. \textbf{11548}, 295--308 (2019)                  
%
\bibitem{km86}
K\v{r}iv\'{a}nek, M., Mor\'{a}vek, J.: NP-hard problems in
hierarchical-tree clustering. Acta informatica.
\textbf{23}, 311--323 (1986)
%
\bibitem{s05}
Schaeffer, S.E.: Graph clustering. Computer Science Review. 
\textbf{1}(1), 27--64 (2005)
%
\bibitem{sst04}
Shamir, R., Sharan, R., Tsur, D.: Cluster graph modification
problems. Discrete Appl. Math. \textbf{144}(1--2), 173--182 (2004)
%
\bibitem{t74}
Tomescu, I.: La reduction minimale d'un graphe \`a une reunion de
cliques. Discrete Math. \textbf{10}(1--2), 173--179 (1974)
%
\bibitem{z64}
Zahn, C.T.: Approximating symmetric relations by equivalence
relations. J. of the Society for Industrial and Applied
Mathematics. \textbf{12}(4), 840--847 (1964)


\end{thebibliography}

\end{document}
